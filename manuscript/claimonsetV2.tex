\section{The Causes of Territorial Claim Onset}
\subsection{Claim types}

Territorial claims can generally be grouped into XXX categories.
- expansionism
Expansionist claims arise when one state desires another state's territory  - economic, military power, national security
- Irredentist claims 
Irridentist claims arise when members of cohesive identity groups (including ethnic, linguistic, and religious groups) wish to unify their fellow group members under the control of a single state.
- secessionism
Post-secessionist claims emerge after a breakaway state successfully establishes independence. These claims include both those by the original state over territory they want to recover and those by the secessionist state over additional territory they wish to incorporate. Since secessionist claims often involve attempts by groups to establish self-rule, post-secessionist claims frequently overlap with irredentist claims. 

Not all claims are actualized
For example, Colombia/peru/nicaragua
First, claims will only emerge when two states value the same piece of territory enough to bear the costs of competition over it. This value, in turn, is determined by the tangible and intangible stakes tied to a piece of land. Second, a host of dyadic factors influence whether both states are willing and able to engage in competition with each other. Third, states must be able to justify their claims on the basis of some normative principle or else face the possibility of domestic or international backlash.


11/173 dyads in Thompson's data had at least one additional issue before a spatial dimension was added
ccode1	ccode2	year	rivdyad
135	155	1879	8
255	300	1866	29
220	255	1871	40
300	640	1878	71
365	710	1963	77
365	740	1905	80
710	750	1959	90
651	666	1956	105
483	620	1972	134
510	553	1967	151
551	553	1967	152



Territorial claims are often closely associated with particular domestic constituencies. 
Ethnicity
Platform
 - Cuban anti-American sentiment
 - N
Legitimacy


\subsection{Dyadic Factors}

%%%%% HISTORICAL OWNERSHIP
% loss aversion
% secessionist state
% colonial powers
	% colonizer tries to regain part of territory
	% colony tries to lose territory
	% do these actually happen - check hensel appendix

%%%%% HISTORY OF OTHER TERRITORIAL DISPUTES
XXX, whether states are or were previously involved in claims over other pieces of territory may increase or decrease the probability of claim onset, depending on how those claims are managed.
When territory has previously changed hands betwen two states,

Whether two states have previously signed agreements over 
Previously signed agreements are unlikely to recur
and whether two states have previously signed agreements delimiting their borders \citep{owsiak2012, huth2009}. % i think owsiak cite needs to be removed
By contrast, states that have successfully negotiated agreements over claims 
hensel2001
States that have previously engaged in miltiary competition over territory are more likely to 


%%%%% CONTIGOUS Colonies
XXX, states that have contiguous colonies (often two major powers) may find themselves competing over colonial land or other issues that arise between their colonial governments (e.g., cross-border rebel groups).

%%%%% DEMOCRACY 
XXX, claims over border territory rarely emerge between two democracies since border settlement often predates the emergence of jointly democratic dyads \citep{gibler2012, gibler2018, owsiak2016d}. 
% TODO Check gibler owsiak 2018


% MUTUAL SECURITY INTERESTS
States that have common security interests, such as mutual alliances or shared rivalies, stand to bear higher costs if issuing a claim leads to diplomatic or military conflict with their opponent.



\subsection{Potential Justifications}

Another important set of factors relates to the potential justifications states can use legitimate a claim to domestic and international actors \citep{burghardt1973, murphy1990, abramson2015}. Although contested territory is (almost) always likely to be valuable, these explanations are limited insofar as they cannot explain why some valuable pieces of land are contested while others are not. In most cases, the presence of valuable territory that both states desire is not sufficient to lead states to issue claims. At the international level, such claims are difficult to justify unless they follow international norms regarding the legitimacy of claims and may create the perception that a state is a purely revisionist or expansionist power. In order to mitigate these costs, states often try to ground claims in principles such as self-determination, irredentism, territorial integrity, and historical sovereignty. International actors tend to view these types of claims as more legitimate and are more likely to side with claimants that have such ties to territory. Historical claims tend to be particularly persuasive to international actors. At the domestic level, such justifications are highly salient and thus make it easier for leaders to rally support for a claim and convince domestic actors to bear the costs associated with it. 

These justifications help explain why states issue claims over some valuable pieces of territory but not others. Although claims may actually be motivated by economic or geopolitical concerns, states still tend to legitimate their claims on the basis of such justifications. As \cite[][p. 544]{murphy1990} notes, ``the contemporary discourse of territorial claim justification is so dominated by the historical argument that most other claims are either left unstated or are offered as support for the historical claim.’’ For example, claims based on historical sovereignty help explain why Ecuador initiated a claim over oil-rich border territory held by Peru but not Colombia \citep{murphy1990}. 



# Theory
\section{Structural and Proximate Causes of Claim Onset}
I begin by assuming that state leaders are primarily responsible for making policy decisions on behalf of the state and that these leaders are rational utility maximizers. 


\subsection{Structural Causes of Claim Onset}

## Intro/issues
From the discussion above, it should be clear that existing literature largely focuses on structural factors. Much of the existing literature focuses on factors that make states willing and able to undertake costly actions to obtain territory. Since the physical attributes of territory are largely immutable, the value of given piece of land does not usually change much over time. These features include characteristics such as arable land, mountainous terrain, rivers and lakes, proximity to the state or its center of power, the presence of natural resources, and access to the ocean.\footnote{Although states may come to value some attributes of territory more or less over time, these changes are likely due to some exogenous shock such as technological change \citep[e.g.,][]{mansbach1981}. For example, the discovery of new resource deposits or the development of technology that makes it possible to extract previously inaccessible resources may increase the value of a particular piece of land. By the same token, the value of resources may diminish over time due to technological change. For example, the value of timber decreased as ironclad warships came to dominate naval warfare.}
	% add more discussion of attributes of territory - ethnicity, symbolic value

## Dyadic Factors
Most of the dyadic variables that influence claim onset fall under the category of structural factors that are fixed or slow-moving over time. Such factors include contiguity, distance between noncontigous states, relative power, major power status, joint democracy, and the presence of transnational identity groups. Historical factors such as prior colonial relationships, a history of military conflict, and the prior exchange of territory are also fixed and influence the underlying opportunity and willingness of states to issue claims.
% Slow changing: interstate relations, systemic factors, balance of capabilities]