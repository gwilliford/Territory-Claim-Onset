# Lit Review
\section{The Causes of Territorial Claim Onset}


\subsection{Claim types}

Territorial claims can generally be grouped into XXX categories.
- expansionism
Expanionist claims arise when one state desires another state's territory  - ecnomic, military power, national security
- Irredentist claims 
Irridentist claims arise when members of cohesive identity groups (including ethnic, linguistic, and religious groups) wish to unify their fellow group members under the control of a single state.
- secessionism
Post-secessionist claims emerge after a breakaway state successfully establishes independence. These claims include both those by the original state over territory they want to recover and those by the secessionist state over additional territory they wish to incorporate. Since secessionist claims often involve attempts by groups to establish self-rule, post-secessionist claims frequently overlap with irredentist claims. 

Not all claims are actualized
For example, Colombia/peru/nicaragua
First, claims will only emerge when two states value the same piece of territory enough to bear the costs of competition over it. This value, in turn, is determined by the tangible and intangible stakes tied to a piece of land. Second, a host of dyadic factors influence whether both states are willing and able to engage in competition with each other. Third, states must be able to justify their claims on the basis of some normative principle or else face the possibility of domestic or international backlash.


% Claims arise over valuable territory / tangible v intangible / description of tangible assets
\subsection{Territorial Stakes}

The emergence of a territorial claim occurs when one state desires territory which the other controls or when two states desire control over the same piece of unclaimed territory. This only happens when there is some piece of territory that both states consider valuable. The literature on contentious issues has established that territory is often tied to tangible and intangible stakes that states prize \citep[e.g., ][]{hensel2001, hensel2008, rosenau1971}. Tangible stakes include a piece of territory's physical characteristics or its contents. These characteristics are often sources of potential economic or strategic value. For example, territory that contains natural resource deposits, arable land, or large population centers represents a potential source of economic gain and military power. Land may also possess strategic value due to it's geographic characteristics, such as the presence of mountain ranges or access to the ocean.

% Intangible assets (nationalism, identity claims)
States may also value land that has connections to identity groups, including national, ethnic, religious, linguistic, or other cultural groups). 
% intangbile
% tangible

Politically powerful groups that have a cohesive sense of identity may push leaders to issue claims over territory that contains members of those groups. These issues tend to underlie claims based on irredentism, secessionism, group unification (e.g., pan-Arabism), and the mistreatment of transnational kin. 

% Border territory
% TODO border territory is particularly salient
% TODO border territory Since borders often create divisions between ethnic groups, these claims often form the basis for postcolonial claims (both between two colonies and between a state and its former colonial ruler). 
% Colonialism
First, contiguous states are much more likely to find themselves in competition over territory. Border territory tends to be more salient than other types of territory due to the fact that it has greater implications for state security and national identity \citep[e.g.,][]{hensel2001}. 
% Stakes
These claims tend to be particularly salient when the contested territory is in close proximity to the state and its homeland, allowing states to easily establish control over the contested territory. % contrast with colonial claims


\subsection{Dyadic Factors}
Several sets of dyadic factors influence whether potential claims involving these issues are likely to emerge between states. 

%%%%% CONTIGUITY AND DISTANCE
First, the spatial relationship between states (i.e., contiguity and distance) has a major influence on the probability that claims emerge, for several reasons. 
A) Only contiguous states compete over valuable border territory.
B) Contiguous states or those in close proximity are also more likely to find themselves competing with each other over both territorial and non-territorial issues. 
	- For example, states that find themselves at odds over regional security issues are more likely to view each other as threats, which in turn increases the value of possessing economically or strategically valuable territory.
	- spatially proximate groups
	- Potential claims over unification or ethnicity are also more likely to arise in states that are in close proximity.
C) Most states are also unable to contest the ownership of distant territory due to the difficulties of projecting force at a distance \citep{boulding1962, lemke2002}.
	% TODO - expalin this more - andy commented w/ question mark

%%%%% MAJOR POWERS AND POWER PARITY
Power dynamics constitute a second set of dyadic characteristics that influence claim onset.
is whether either or both states are a major power. Compared to other states, major powers have an easier time projecting force at a distance (CITATION). This makes it easier for them to contest territory held by far-flung states and to establish overseas colonies.

Other dyadic factors influence the opportunity for states to contend with each other over territory. One important factors is whether potential challengers have the power to persuade their opponents to relinquish control of the territory. Even when claims do not explicitly involve militarized interactions, the potential use of force to acquire territory is often an underlying factor that influences each state’s incentives to obtain or relinquish contested territory \citep{huth2009, waltz2010}. Claims are thus more likely to emerge when the challenger has a military advantage over the target and less likely to emerge when the target has military allies \citep{huth2009}. 
Dyads that contain one major power vs 2
% TODO - what about challenger allies
% TODO - but norm? previous paragraph 

%%%%% HISTORY OF MILTIARY COMPETITION
A XXX set of dyadic factors involve the extent to which states regard each other as a threat to their interests. States that view each other as threats have a greater incentive to obtain economically or strategically valuable territory \citep{colaresi2007, goertz2001}. This is likely to emerge when states view each other as particularly hostile or have a history of miltiarized disputes over other issue claims. This is particularly true when disputes have emerged over prior territorial claims, which tend to elicit the use of power politics and contribute to feelings of threat and enmity \citep{hensel2001, vasquez2009}.


%%%%% HISTORICAL OWNERSHIP
% loss aversion
% secessionist state
% colonial powers
	% colonizer tries to regain part of territory
	% colony tries to lose territory
	% do these actually happen - check hensel appendix


%%%%% HISTORY OF OTHER TERRITORIAL DISPUTES
XXX, whether states are or were previously involved in claims over other pieces of territory may increase or decrease the probability of claim onset, depending on how those claims are managed.
When territory has previously changed hands betwen two states,

Whether two states have previously signed agreements over 
Previously signed agreements are unlikely to recur
and whether two states have previously signed agreements delimiting their borders \citep{owsiak2012, huth2009}. % i think owsiak cite needs to be removed
By contrast, states that have successfully negotiated agreements over claims 
hensel2001
States that have previously engaged in miltiary competition over territory are more likely to 


%%%%% CONTIGOUS Colonies
XXX, states that have contiguous colonies (often two major powers) may find themselves competing over colonial land or other issues that arise between their colonial governments (e.g., cross-border rebel groups).

%%%%% DEMOCRACY 
XXX, claims over border territory rarely emerge between two democracies since border settlement often predates the emergence of jointly democratic dyads \citep{gibler2012, gibler2018, owsiak2016d}. 
% TODO Check gibler owsiak 2018


% MUTUAL SECURITY INTERESTS
States that have common security interests, such as mutual alliances or shared rivalies, stand to bear higher costs if issuing a claim leads to diplomatic or military conflict with their opponent.



\subsection{Potential Justifications}

Another important set of factors relates to the potential justifications states can use legitimate a claim to domestic and international actors \citep{burghardt1973, murphy1990, abramson2015}. Although contested territory is (almost) always likely to be valuable, these explanations are limited insofar as they cannot explain why some valuable pieces of land are contested while others are not. In most cases, the presence of valuable territory that both states desire is not sufficient to lead states to issue claims. At the international level, such claims are difficult to justify unless they follow international norms regarding the legitimacy of claims and may create the perception that a state is a purely revisionist or expansionist power. In order to mitigate these costs, states often try to ground claims in principles such as self-determination, irredentism, territorial integrity, and historical sovereignty. International actors tend to view these types of claims as more legitimate and are more likely to side with claimants that have such ties to territory. Historical claims tend to be particularly persuasive to international actors. At the domestic level, such justifications are highly salient and thus make it easier for leaders to rally support for a claim and convince domestic actors to bear the costs associated with it. 

These justifications help explain why states issue claims over some valuable pieces of territory but not others. Although claims may actually be motivated by economic or geopolitical concerns, states still tend to legitimate their claims on the basis of such justifications. As \cite[][p. 544]{murphy1990} notes, ``the contemporary discourse of territorial claim justification is so dominated by the historical argument that most other claims are either left unstated or are offered as support for the historical claim.’’ For example, claims based on historical sovereignty help explain why Ecuador initiated a claim over oil-rich border territory held by Peru but not Colombia \citep{murphy1990}. 



# Theory
\section{Structural and Proximate Causes of Claim Onset}
I begin by assuming that state leaders are primarily responsible for making policy decisions on behalf of the state and that these leaders are rational utility maximizers. 


\subsection{Structural Causes of Claim Onset}

## Intro/issues
From the discussion above, it should be clear that existing literature largely focuses on structural factors. Much of the existing literature focuses on factors that make states willing and able to undertake costly actions to obtain territory. Since the physical attributes of territory are largely immutable, the value of given piece of land does not usually change much over time. These features include characteristics such as arable land, mountainous terrain, rivers and lakes, proximity to the state or its center of power, the presence of natural resources, and access to the ocean.\footnote{Although states may come to value some attributes of territory more or less over time, these changes are likely due to some exogenous shock such as technological change \citep[e.g.,][]{mansbach1981}. For example, the discovery of new resource deposits or the development of technology that makes it possible to extract previously inaccessible resources may increase the value of a particular piece of land. By the same token, the value of resources may diminish over time due to technological change. For example, the value of timber decreased as ironclad warships came to dominate naval warfare.}
	% add more discussion of attributes of territory - ethnicity, symbolic value

## Dyadic Factors
Most of the dyadic variables that influence claim onset fall under the category of structural factors that are fixed or slow-moving over time. Such factors include contiguity, distance between noncontigous states, relative power, major power status, joint democracy, and the presence of transnational identity groups. Historical factors such as prior colonial relationships, a history of military conflict, and the prior exchange of territory are also fixed and influence the underlying opportunity and willingness of states to issue claims.
% Slow changing: interstate relations, systemic factors, balance of capabilities]



\subsection{Proximate Causes of Claim Onset}

## Domestic Proximate Causes


## International Proximate Causes

