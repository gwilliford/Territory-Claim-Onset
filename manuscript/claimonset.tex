\begin{document}
\chapter{The Proximate Causes of Territorial Disputes}
\pagebreak

\section{Introduction}

% Cases

What are the proximate causes that lead territorial claims to emerge between states? States often justify claims to another state’s territory on the basis of principles such as self-determination, irredentism, and historical sovereignty. Yet in many cases, leaders do not immediately issue claims once the potential justification to do so arises. For example, although Cuba was coerced into conceding territory to the United States in 1903, it was nearly six decades before the state challenged the U.S. occupation of this territory. This changed in 1960 following the Cuban Revolution. What explains the sudden reversal of Cuban policy? The answer lies in a change of the preferences of the state’s winning coalition. In contrast to the pro-American military government that presided under Batista, the coalition of supporters that brought Castro to power harbored anti-American sentiments \citep{morley1982, wright2000}. Similar examples involve claims between Cambodia and Vietnam, Nigeria and Cameroon, Nicaragua and Colombia, Germany and Austria, and Tanzania and Uganda. In each of these cases, challengers had a number of potential motivations and justifications for issuing a claim against the other state. 

% Puzzle

The examples above produce a puzzle: empirically, most states rarely issue claims against other states at the first opportunity to do so (e.g., the birth of a dyad or the emergence of the stakes that motivate a claim). For example, although claims often have their roots in the events surrounding the birth of a state, eighty-five percent of claims in the Issue Correlates of War Territorial claims dataset begin after state birth \citep{frederick2017}. Why do states often wait to issue claims for years or decades after a potential motivation or justification for issuing a claim emerges?

As discussed at length in the previous chapter, issuing new issue claims against other states often carries substantial diplomatic, economic, and military costs \citep[e.g.,][]{akcinaroglu2014, diez2006, fravel2008, gibler2012, goertz2016, lee2012, pressman2008, simmons2005, vasquez2001, vasquez2009}. As such, leaders will only initiate claims when the expected utility of doing so exceeds that of maintaining the status quo. The decision to initiate a new claim thus implies that there have been changes in the factors that determine states’ opportunity and willingness to do so \citep{most1989}. 

% Territorial claims are costly to maintain. Since they often bring with them the threat of militarized disputes and costly rivalries, maintaining a claim to territory often requires a substantial investment in the military and preventing leaders from pursuing other valuable domestic and foreign policy goals \citep{vasquez2001, vasquez2009}. Claims can also lead to economic costs in the form of decreased trade and foreign direct investment among the disputants \citep{lee2012, simmons2005}. States may also face costs imposed by third parties [citation].

% Lit review

Existing studies on claim onset have largely focused on the underlying or structural causes of claims. These factors include the issues that motivate claims, the dyadic relationship between disputants, and international norms \citep[e.g.,][]{abramson2015, burghardt1973, carter2011, englebert2002, goemans2016, huth2009, murphy1990, schultz2015}. Although these factors undoubtedly influence whether a claim is likely to emerge between two states, each tends to be fixed or change very slowly over time. However, existing research neglects the issue of the timing of claim onset. In many cases states who could take advantage of such justifications go for decades without issuing claims over desirable territory, only to reverse course seemingly overnight. 

% (e.g., economic growth, natural security, and group identity),

% (e.g., self-determination, irredentism, and historical sovereignty)

% (e.g., contiguity, major power status, and a history of militarized conflict)

% Proximate Causes

With respect to proximate causes, I focus on two factors that have the potential to change the opportunity and willingness of states to issue claims. First, I argue that changes in the preferences of a state’s leaders are often a cause of claim onset. This is most likely when there is turnover in the leader herself or a change in the coalition of political actors that form a leader’s base of power. Second, I examine whether changes in the balance of military capabilities between two disputants are associated with claim onset. Theoretically shifts that increase the power of a challenger state relative to a potential target should increase their bargaining power and make them more likely to issue costly claims. 

% Cure Models	

This chapter demonstrates the utility of cure models in two ways. First, it illustrates how cure models can be used to distinguish how variables influence the probability and timing of claim onset. The cure model provides a means of distinguishing between the theoretical effects of variables that are underlying causes of a phenomenon and those that are the proximate causes of events. By explicitly distinguishing between structural and proximate causes, my theory provides guidance as to which factors affect the potential for a claim to emerge between states and the timing of claim onset. In doing so, it alleviates potential bias and provides for better theoretical insights.

Second, the chapter demonstrates the utility of cure models for the analysis of rare events in international relations. Many studies involve the analysis of hundreds of thousands or millions of dyads, only a fraction of which are likely to experience the event of interest. As discussed in the methods chapter, the inclusion of a large number of irrelevant cases has the potential to create biased results, particularly when there is a large imbalance in the number of events and nonevents. In an attempt to avoid these problems, many scholars rely on the use of case selection strategies, e.g., limiting the dataset to politically relevant dyads. However, the use of such strategies has the potential to create selection bias both by excluding relevant cases (e.g., disputes between noncontiguous, non-major powers) and including irrelevant cases (e.g., conflicts between two land-locked and distant weak powers). The cure model avoids this by explicitly modeling this process and reweighting the estimates of the hazard model accordingly. 

% Findings

I test my argument on a global sample using data from the Issue Correlates of War Territorial Change dataset to identify the beginning of territorial claims. Controlling for the structural characteristics that create the opportunity for claims to emerge, I demonstrate that changes in a country’s leadership or winning coalition are associated with an increase in the hazard of claim onset. I also find that militarized interactions and periods of rapid systemic changes, such as the World Wars, increase the probability of claim initiation. I also demonstrate that using a cure model to predict the potential for claim onset produces different substantive inferences than the use of a Cox model on subsamples of politically relevant dyads and rivalry. This provides a more nuanced method of dealing with the inclusion of extraneous variables that avoids the selection effect produced by using a subsample of the data. 



\section{Structural and Proximate Causes of Claim Onset}

% Rationality and opp/willingness



Since initiating claims is costly, challengers will only initiate claims when the expected utility of doing so exceeds that of maintaining the status quo and when they perceive that they have the opportunity to do so \citep{most1989}. In the context of territorial claims, willingness is a function of the value of the territory itself as well as any costs that the challenger may accrue by issuing a claim. Opportunity refers to the possibility that a state can bring about a change in the status quo distribution of territory by taking actions that convince their opponent to relinquish their own claim to it (e.g., making threats or offering side-payments). Regardless of the costs that a challenger is willing to bear, they will not choose to do so if these efforts are futile.

% Structural v proximate causes

Understanding why territorial claims emerge requires answering two questions. First, under what conditions is it possible that a claim will emerge between two states? Second, what factors explain when leaders decide to initiate claims? To answer these questions it is necessary to distinguish between the underlying structural factors that create the potential for claims between states and the proximate causes that lead leaders to initiate them \citep[see, e.g.,][]{mansbach1981, stinnett2001, vasquez2009}.

% Structural

The answer to the first question lies in the underlying or structural causes that creates the potential for states to find themselves in competition over the same piece of territory. In order for a claim to emerge, two conditions must hold. First, there must be a piece of territory held by the target that the challenger values highly to undertake costly actions to obtain it. Second, challenger states must have the opportunity to pursue actions capable of persuading the other state to relinquish their claim to the territory. Structural factors such as the value of territory or long-standing characteristics of two states’ relationship are outside of states’ immediate control and are therefore either fixed or change slowly over time. Such factors may include characteristics of the state, the dyadic relationship, or the international system. In the absence of structural causes, leaders have no incentive to initiate costly claims even in the presence of proximate causes. 

% Rather than emerging from states’ immediate behavior, structural factors describe fundamental features of the relationship between two states and the environmental context in which they operate.

% Proximate

The answer to the second question lies with the proximate causes that push leaders to abandon the status quo in favor of initiating a claim. The choice to initiate new claims requires that the expected utility of doing so has increased to the point that it exceeds that of maintaining the status quo. In order for this to occur, there must either be changes in the opportunity or willingness for states to pursue these claims.\footnote{\citet{mansbach1981} refer to these as behavioral causes of issue emergence.} Changes in willingness require a shift in the value that leaders assign to initiating a claim, while changes in opportunity occur when changes in the dyadic relationship or changes in the international system enhance the capability of challengers to pursue claims. By their nature, proximate causes encompass dynamic variables or short-lived ``shocks’’ that lead to fundamental changes in the status quo \citep[see, e.g.,][]{goertz1995}. In the absence of proximate causes, leaders do not have an incentive to deviate from status quo policy regardless of whether the potential for a claim to emerge exists.



\subsection{Proximate Causes of Claim Onset}

% Opportunity/Capability Change

As noted above, the proximate causes of claim onset involve changes in the opportunity and willingness of states to do so. The emergence of new opportunities is typically caused by changes in the dyadic relationship between states or in the international environment that remove constraints that may prevent states from pursuing certain courses of action. The most obvious way that new opportunities arise occurs when one there are changes in the relative power of two states. As discussed above, relative power determines the extent to which one may be able to threaten the other with the use of force as a means of acquiring territory, and therefore determines whether challengers have the potential to coerce their opponent into making concessions. As a potential challenger state becomes more powerful relative to a potential target, their ability to impose costs on their opponents and potentially seize a piece of territory by force increases \citep[e.g.,][]{lemke2002, organski1981}. Increases in power can also enhance a state’s ability to offer compensatory side payments to the other state in exchange for the territory in question. Given this newfound bargaining leverage, these states may seek revision of the status quo \citep{fearon1995, powell2006}. Capability changes may be particularly relevant when deciding to renew old claims. If one state conceded the territory to their opponent in the past because they were weaker, an increase in their capabilities relative to their opponent is likely play a role in their decision to renew their claim.

% International willingness/MIDs

Absent changes in the environment in which states operate, behavioral changes typically occur when there is a change in the preferences of a state’s leadership. 

This may occur due to the occurrence of events in the international system that alter the dyadic relationship between states or due to changes in the domestic level. At the international level, one factor that may influence states’ willingness to issue claims is the occurrence of conflictual interactions between states. The occurrence of militarized disputes between states alters their expectations regarding the behavior of other states and thereby increases the salience of potential claims against that state’s territory \citep{hensel2001, colaresi2007, vasquez2009}. When this occurs the expected utility of issuing a claim increases. This may be particularly relevant when states compete over economically or strategically valuable territory due to commitment problems \citep{fearon1995}. As hostility between states increases the value of obtaining such territory increases. MIDs and crises may also lead leaders to reprioritize certain agenda items over others as these become more pressing issues. This may bring security concerns to the forefront of the foreign policy agenda and thereby increase the utility of obtaining territory \citep{wood1998}. 

% Domestic willingness/leader/coalitionchange

At the domestic level, one obvious cause of actor’s preferences is a change in the leadership or winning coalition of a state. Although actor’s preferences may change over time, radical shifts in their preferences regarding foreign policy are unlikely to occur absent major changes in the international or domestic environment. As such, large shifts in the direction of state policy (e.g., issuing a territorial claim) are most likely to occur due to changes in the domestic actors that control the levers of power \citep{cox1982, mattes2015}.

Two actors are particularly important in this regard: a state’s leader and the winning coalition that support them. As discussed in the previous chapter, political leaders are ultimately responsible for determining what policies their state pursues. However, leaders are ultimately beholden to a winning coalition, a group of political supporters who has the power to remove leaders if they so choose \citep{bdm2003}. Although leadership and coalition changes often coincide, this is not necessarily the case. The same winning coalition may remain in power while selecting a new leader, as in the election of a new leader or a coup within the ranks of the military. As such, it is worth distinguishing between the effects of leadership change and those of coalition change, although the greatest shift in preferences is most likely to occur when both coincide.

% Leaders Change

Even in the absence of changes in the winning coalition, changes in leadership can have substantial effects on foreign policy. All individuals hold different attitudes, beliefs, motives, perspectives, and predispositions and differ in terms of traits such as risk-aversion, cognitive complexity, and personality characteristics. All of these factors influence leaders’ foreign policy goals and their views on the utility of certain actions such as the use of force \citep[see, e.g.,][]{etheredge1978, guetzkow1981, haas1974, hermann1974, hoffmann1970, kahneman1979, wittkopf1983}. This is particularly true in the context of territorial claims, as leaders’ proclivities for hawkish or dovish policies determine whether states engage in militarized or peaceful conflict management attempts or maintain the status quo \citep[e.g.,][]{chiozza2003, colaresi2004, colaresi2005, vasquez2009}. As such, it stands to reason that leaders would have different preferences in regard to whether or not to initiate claims. 

% Leaders may also have unique social locations that allow them to benefit more domestically from certain courses of actions than other leaders.

As a result, changes in leadership may bring to a power an individual whose preferences with respect to claim initiation differ from their predecessor’s, even when leaders are not removed from office by the winning coalition (for example, when leaders leave office due to term limits). However, although any new leader will always have different preferences than their predecessor over at least some issues, coalitions who replace their own leadership often do so due to the fact that the incumbent’s preferences and actions do not align with their own. Because territory is highly salient to domestic actors, potential challengers may be able to mobilize support within the winning coalition by promising to take actions that better satisfy their preferences \citep{colaresi2004, colaresi2005, vasquez2009}. 

% Example: Nigeria-Cameroon	

One example of the influence of leadership transitions involves a dispute between Nigeria and Cameroon over the Bakassi Peninsula. Upon gaining independence in 1960, the two states inherited a long-standing border dispute between their respective colonizers, Britain and Germany. Although these two countries outlined a general border agreement, it did not fully demarcate the boundary and disagreements over the exact location of the border persisted. The Bakassi Peninsula contains valuable natural resources, constitutes a strategic location, and is the subject of claims based on ethnic identity, historical sovereignty, and claims to homeland status. In addition, both states had claims to the territory based on identity, historical ownership, and homeland status. Nonetheless, the boundary issue between the two countries remained dormant for many years, with neither issuing explicit claims to the areas where sovereignty was not well-established. This changed following a coup within Nigeria’s military-led government by Ramat Muhammad in 1975. Although the military remained the source of political power, the change in leadership brought about a shift in Nigeria’s policy regarding the border.

% Example: Cambodia Vietnam

Another example of this involves a claim between Cambodia and Vietnam. Following the election of Norodom Ranariddh as the new head of Prime Minister of Cambodia in 1993, Cambodia renewed its claims to various location along its border with Vietnam. Although the winning coalition remained unchanged (i.e., the National United Front for an Independent, Neutral, Peaceful and Cooperative Cambodia), public opinion favored the resolution of these dormant claims with Vietnam \citep{amer1997}.

% Coalition Change

In addition to changes in a state’s leader, states often experience changes in the ruling coalition itself. Since coalition changes result in a change in the base of support for political power changes, they often entail larger shifts in preferences \citep{bdm2003, cox1982, mattes2015, mattes2016}. Just as challengers within a coalition may use territorial issues to elicit support, opposition leaders outside the coalition may mobilize supporters by promising to pursue a claim. This may occur even if the new coalition retains an old leader since the new leader has an incentive to alter her behavior to fit their preferences. However, this scenario implies that there is at least some continuity in the preferences of the two coalitions with respect to their preferences regarding highly salient issues. All else equal, coalition changes that produce changes in leadership are more likely to lead to changes in status quo policy. 

% By the same token, leadership changes are more likely to lead to substantial changes in policy when they occur in conjunction with coalition changes. Although new leaders may be replaced by the winning coalition because they do not follow the winning coalition’s preferences, there should still be a greater continuity in preferences when these leaders have the same base of support.

% Example: Kagera Salient

One example of this involves a claim between Uganda and Tanzania. Following the military coup that brought Idi Amin to power in 1971, Uganda issued a claim to the Kagera Salient, a region along its border then under the control of neighboring Tanzania. Uganda’s claim to the Kagera Salient could potentially be justified on multiple bases, including historical agreements, ethnic unification, and geographic considerations (Amin argued that extending the border to the Kagera River provided a more natural division of territory). However, the fact that these claims went unutilized until Amin took power suggests that his own personal motivations were an important factor in his decision to issue a claim. The border partitioned ethnic groups to which Amin belonged. Amin also had personal motivations for claiming control of the Kagera Salient which served as a base for opposition rebels \citep{valeriano2011}.

% Example: Nicaragua-Colombia

Another example involves the dispute between Nicaragua and Colombia over the San Andreas and Providencia archipelago. After signing a treaty resolving the dispute in 1928, the dispute remained dormant for over 60 years. This agreement persisted until 1979 when the Sandanista government took power and declared the old agreement null due to the fact that it had been signed under pressure from the U.S. Likewise, Nigeria renewed its claims to the banks of Quita Suensueno, Serrana, and Roncador. Additional examples include Germany’s decision to issue a claim to Austrian territory following the rise of the Nazi party in 1933 and Cuba’s decision to contest the American military presence on the island following the rise of Castro’s regime.

\section{Research Design}

%   Data/Cure Model	

I test my argument on a sample of all dyads in the international system from 1816-2001. The dependent variable of interest is the onset of territorial claims, which I measure using data from the Issue Correlates of War (ICOW) Territorial Claims Dataset, version 1.01 \citep{frederick2017}. The dataset contains data on territorial claims between all states in the international system from 1816 to 2001. ICOW codes the onset of a claim when there is a statement by a high-level official representative of the state that indicates that they regard a piece of territory as belonging to their state. Importantly, ICOW does not code the existence of a claim when opposition leaders or other societal groups make claims to land but the state does not. 

My overall dataset includes all dyads in the international system from 1816-2001. Dyads enter the dataset in the first year in which both states become independent. I code the onset of a claim using a dummy variable for whether a new ICOW claim began between the two in a given year. Since most dyads never experience claim onset, the use of a cure model to distinguish between those dyads that are at risk and those that are not is necessary to obtain consistent estimates of the influence of variables that affect the timing of claim onset. Since my provides an explicit distinction between issues that influence the underlying risk of an event and those that influence the timing of an event, the use of a cure model is ideal for modelling the process that generates claims and provides greater theoretical insights by indicating \textit{how} each variable influences claim onset. I use the cure component of the model to model the structural factors that influence claim onset and the hazard component to model the proximate causes. I refer to these as the “structural model” and “proximate model” respectively. 

% Repeated events?

% The baseline hazard captures the time since the beginning of the life of a dyad until the onset of a territorial claim.

% PRD

One advantage of using a cure model in this situation is that it can account for the varying levels of risk that underlie the selection process without resorting to the use of selection criteria such as the use of politically relevant dyads. These methods solve the problem of “excess zeroes” by removing cases that are supposedly not at risk from the dataset entirely. However, doing so requires that scholars can easily distinguish between those cases that are at risk and those that are not. However, since conflict processes are probabilistic, no single variable (or set of variables) can determine which cases are at risk and which are not. As a result, limiting the sample to politically relevant dyads often excludes many dyads that do experience conflict and includes many that are highly unlikely to do so \citep{lemke2001}. Both of these outcomes have the potential to produce biased results. On the one hand, the omission of relevant cases constitutes selection bias. On the other hand, as noted in the methods chapter, the inclusion of irrelevant cases tends to bias the hazard coefficients towards zero. Using the cure model allows me to account for the many and varied structural factors that account for the differential risk between cases alleviates these sources of bias with it without resorting to heavy-handed case selection techniques.

\subsection{Structural Model}

% Stakes

The structural model accounts for fundamental characteristics of the dyadic relationship that determine whether two states have the opportunity and willingness to compete over the same piece of land. To model the probability that a piece of territory that both states desire emerges, I include several measures. To capture the spatial relationship between states, I include the distance between capital cities using data from \citet{bennett2000, stinnett2002}. Because it is heavily skewed, I log this variable also account for the potential for clashes between two colonial powers using the Correlates of War Colonial Contiguity dataset \citep{cow2020}. I also include measures of whether one state is a former colony of the other \citep{hensel2018} and whether there has previously been an transfer of territory between the two states \citep{tir1998}. To account for the history of conflict between two states, I include a dummy variable for whether two states are involved in a rivalry in a given year \citep{colaresi2007}. Finally, I measure whether two states are both democracies using an indicator for whether both states have a Polity score above 5 \citep{marshall2013}.

% . I account for potential claims based on ethnicity using a dummy variable for whether two states share an ethnic group according to the Transnational Ethnic Kin dataset \citep{ruegger2018, vogt2015}.

% Alternative controls for spatial relationships: region

% Controls that could be included for contiguous states: Resources, Strategic location, Population, Length of border, settled border

% Other dyadic factors: History/ Power/ Mutual Interests/Negative Externalities/Joint Democracy

I also include several measures designed to account for dyadic factors that influence the opportunity for claims to emerge. To account for differences in power between two states I include variables for whether a dyad is composed of two minor powers, one major power, or two major powers, with two minor powers left out as the reference category \citep{cow2017}. To account for whether two states have mutual interests, I include a dummy variable for whether two states share a defensive alliance \citep{gibler2004}. 

% Alternative controls for mutual intersts: mutual rivals, economic interdependence

% and the number of previous militarized interstate disputes between the two \citep{gibler2016}

% and a measure of the number of intergovernmental organizations to which each state belongs \citep{peverhouse2019}.

\subsection{Proximate Model}

% International causes

To capture changes in capabilities, I use the percent change in the capability ratio of two states in a given year \citep{singer1987}. The capability ratio variable ranges from 0 to 0.5, where lower values indicate greater asymmetry in the power between disputants. Positive values of the percent change variable thus indicate a movement towards relative parity. To assess the influence of militarized disputes on the probability of claim onset, I code a variable for the onset of a MID in the previous year \citep{gibler2016}.

Other potential proximate causes of territorial claims are ``shocks’’ at the international level. \citet{goertz1995} argue that major changes in the international system can bring about fundamental changes in the relationship between states and therefore influence the probability of rivalry onset and termination. By the same token, these may also alter the relationship between states in such a way as to change the opportunity and willingness for states to initiate territorial claims. At the international level, \citet{goertz2005} identify four periods in which major changes in the international system occurred: the period surrounding German and Italian unification (1859-1877), World War I, World War II, and the End of the Cold War.\footnote{I code World War II as beginning in 1937 to account for the beginning of the war in Asia and the numerous territorial claims that emerge as a result}. I account for each of these using a dummy variable for whether each year occurred during a shock or in the 5 years afterwards. 

% additional vars: rivalry termination

% Systemic vars: third party rivalry termination, external MIDs

% Domestic Causes

To determine how leadership changes influence claim onset I use the Archigos dataset \citep{goemans2009}. I code a dummy variable equal to 1 if either state experienced a leadership change in a given year. Since leaders often initiate claims shortly after taking office, it is important to measure whether a leadership transition occurs in year $t$. However, the fact that the data are aggregated at the yearly level creates the possibility for simultaneity bias. From a theoretical perspective, there is little reason to suspect that claim onset will lead to leadership transitions in the same year. Contrary to making concessions with respect to a territorial claim, issuing a claim should not be costly enough to lead domestic audiences to immediately remove a leader from power. To address this, I estimate separate models using the contemporary value of leadership change as well as its one-year lag.

% This may occur due to reverse causality (i.e., issuing a claim leads to leadership transitions) or merely because both happen to coincide in the same year.

To assess how changes in the winning coalition influence claim onset, I use the Change in Source of Leader Support (CHISOLS) dataset \citep{mattes2016}. CHISOLS identifies when the societal groups that form the primary base of support for political leaders changes. I code a dummy variable for whether the winning coalition in at least one state within a dyad occurs within a given year. As with leadership change, I estimate models with both the contempora. For models examining domestic factors, I also include variables for whether either state experiences a civil war, undergoes regime change, or gains independence in a given year \citep{ cow2017, gleditsch2002a, goertz1995, marshall2013, mattes2016}.

% additional domestic controls: Government crises, riots, leaders support




\section{Analysis}
This section proceeds in 4 parts. First, I begin by analyzing the international proximate causes of territorial claim onset. Second, I turn to the models discussing the domestic proximate causes. Third, I demonstrate that the use of cure models produces different results than simply using a cox model with the case selection devices of politically relevant dyads and rivalry. Finally, I briefly discuss the results of the structural model and how those findings relate to the current literature.

\begin{comment}
add titles
& \multicolumn{2}{c}{\underline{Model 1}} & \multicolumn{2}{c}{\underline{Model 2}} \\
& Structural & Proximate & Structural & Proximate \\ \hline
"\multicolumn{2}{c}{Model 1}", "&", "\multicolumn{2}{c}{Model 2}", " &")
Replace incidence
Replace zeroes
htpb
\end{comment}

\subsection{International Proximate Causes of Territorial Claims}

% latex table generated in R 3.6.2 by xtable 1.8-4 package
% Wed Oct 14 16:58:50 2020
\begin{table}[htpb]
	\caption{International Proximate Causes of Territorial Claim Onset \label{tab_i}}
	\centering
	\begin{tabular}{lllll}
		\hline
		& \multicolumn{2}{c}{\underline{Model 1}} & \multicolumn{2}{c}{\underline{Model 2}} \\
		& Structural & Proximate & Structural & Proximate \\ \hline
		Percent Change Capabilities$_{t}$ &  & 0.011 &  &  \\ 
		&  & (0.091)  &  &  \\ 
		Percent Change Capabilities$_{t-1}$ &  &  &  & 0.01 \\ 
		&  &  &  & (0.083)  \\ 
		MID$_{t}$ &  & 0.417 &  &  \\ 
		&  & (0.054)*** &  &  \\ 
		MID$_{t-1}$ &  &  &  & 0.134 \\ 
		&  &  &  & (0.062)* \\ 
		System Change (1859-1877) &  & 0.433 &  & 0.386 \\ 
		&  & (0.093)*** &  & (0.102)*** \\ 
		World War I &  & 0.269 &  & 0.31 \\ 
		&  & (0.052)*** &  & (0.058)*** \\ 
		World War II &  & 0.351 &  & 0.352 \\ 
		&  & (0.062)*** &  & (0.051)*** \\ 
		Coldwar Termination &  & 0.292 &  & 0.268 \\ 
		&  & (0.152). &  & (0.151) \\ 
		Distance & -0.178 &  & -0.177 &  \\ 
		& (0.026)*** &  & (0.023)*** &  \\ 
		Previous Territorial Change & 0.319 &  & 0.298 &  \\ 
		& (0.135)* &  & (0.154) &  \\ 
		Former Colony & 0.668 &  & 0.659 &  \\ 
		& (0.183)*** &  & (0.179)*** &  \\ 
		Colonial Contiguity & 2.208 &  & 2.068 &  \\ 
		& (0.266)*** &  & (0.215)*** &  \\ 
		Major-Minor Dyad & 1.363 &  & 1.521 &  \\ 
		& (0.118)*** &  & (0.129)*** &  \\ 
		Major-Major Dyad & 0.933 &  & 1.084 &  \\ 
		& (0.276)*** &  & (0.16)*** &  \\ 
		Defensive Alliance & -0.079 &  & 0.032 &  \\ 
		& (0.09)  &  & (0.093)  &  \\ 
		Joint Democracy & -0.667 &  & -0.817 &  \\ 
		& (0.219)** &  & (0.136)*** &  \\ 
		Rivalry & 1.852 &  & 1.971 &  \\ 
		& (0.153)*** &  & (0.15)*** &  \\ 
		Intercept & -7.563 &  & -7.644 &  \\ 
		& (0.262)*** &  & (0.208)*** &  \\ 
		Number of Observations & 695525 &  & 676249 &  \\ 
		Number of Failures & 441 &  & 396 &  \\ 
		\hline
	\end{tabular}
\end{table}

Table \ref{tab_i} presents the results of two models examining the proximate causes of territorial disputes at the international level. Model 1 includes variables for year $t$, while Model 2 contains variables for $t-1$. Several variables are worth examining in detail. First, the occurrence of MIDs in either the current year or the year before are associated with an increase in the hazard rate of 0.417 and 0.134 respectively. Exponentiating those coefficients provides hazard ratios of 1.517 and 1.14 respectively, indicating that contemporaneous MIDs are associated with a 52.7 percent increase in the hazard rate, while MIDs in the prior year increase the hazard rate by 14.3 percent.

Second, three of the four ``shock'' variables are positive and significant, indicating that periods of abrupt change in the international system are associated with a higher hazard rate. Specifically, the period from 1859-1877 has a coefficient of 0.433 in Model 1 and 0.386 in Model 2, which are associated with an increase in the hazard rate of 1.542 and 1.471 respectively. Thus, the period surrounding German and Italian unification is associated with a roughly 50 percent higher hazard rate. 

World War I is also associated with an increased hazard. The estimated coefficients and hazard ratios are 0.269 (HR: 1.319) and 0.310 (HR: 1.36) respectively. This indicates that the onset of territorial claims is more likely during World War I and for several years afterwards. Likewise, the estimated coefficients for World War II are 0.351 (HR: 1.420) and 0.352 (HR: 1.422), indicating that World War II is associated with a roughly 40 percent increase in the hazard rate. Taken together, three of the four shock variables are significant, indicating that periods of rapid change when shifts in preferences and opportunities are highly likely increase the probability of claim onset.


\subsection{Domestic Proximate Causes of Territorial Claims}
% LEADER CHANGE ---------------------------------------------------------------------------------

% latex table generated in R 3.6.2 by xtable 1.8-4 package
% Wed Oct 14 19:43:31 2020
\begin{table}[htpb]
	\caption{Leadership Change and Territorial Claim Onset \label{tab_l}}
	\centering
	\begin{tabular}{lllll}
		\hline
		& \multicolumn{2}{c}{\underline{Model 1}} & \multicolumn{2}{c}{\underline{Model 2}} \\
		& Structural & Proximate                  & Structural & Proximate                  \\ \hline
		Leadership Change$_{t}$     &            & 0.124                      &            &                            \\
		&            & (0.053)*                   &            &                            \\
		Leadership Change$_{t-1}$   &            &                            &            & 0.11                       \\
		&            &                            &            & (0.045)*                   \\
		Independence$_t$            &            & 0.37                       &            &                            \\
		&            & (0.057)***                 &            &                            \\
		Independence$_{t-1}$        &            &                            &            & 0.167                      \\
		&            &                            &            & (0.1)                     \\
		Regime Transition$_t$       &            & 0.154                      &            &                            \\
		&            & (0.081)                   &            &                            \\
		Regime Transition$_{t-1}$   &            &                            &            & 0.042                      \\
		&            &                            &            & (0.091)                    \\
		Distance                    & -0.255     &                            & -0.242     &                            \\
		& (0.037)*** &                            & (0.026)*** &                            \\
		Previous Territorial Change & 0.091      &                            & -0.012     &                            \\
		& (0.154)    &                            & (0.141)    &                            \\
		Former Colony               & 0.74       &                            & 0.723      &                            \\
		& (0.191)*** &                            & (0.183)*** &                            \\
		Colonial Contiguity         & 2.263      &                            & 2.447      &                            \\
		& (0.306)*** &                            & (0.236)*** &                            \\
		Major-Minor Dyad            & 1.364      &                            & 1.443      &                            \\
		& (0.099)*** &                            & (0.113)*** &                            \\
		Major-Major Dyad            & 1.628      &                            & 1.848      &                            \\
		& (0.349)*** &                            & (0.389)*** &                            \\
		Defensive Alliance          & -0.215     &                            & -0.152     &                            \\
		& (0.097)*   &                            & (0.103)    &                            \\
		Joint Democracy             & -0.51      &                            & -0.631     &                            \\
		& (0.206)*   &                            & (0.19)***  &                            \\
		Rivalry                     & 2.076      &                            & 2          &                            \\
		& (0.11)***  &                            & (0.137)*** &                            \\
		Intercept                   & -7.299     &                            & -7.556     &                            \\
		& (0.329)*** &                            & (0.231)*** &                            \\
		Number of Observations      & 661747     &                           & 659438     &                          \\
		Number of Failures          & 378        &                           & 341        &                          \\ \hline
	\end{tabular}
\end{table}

Table \ref{l_tab} presents the results of models of the influence of leadership transitions on claim onset. Model 1 includes a measure of whether leadership changes occur in year $t$. The estimated coefficient is positive and significant, indicating that leadership changes increase hazard of claim onset. The estimated hazard coefficient is 0.124 (HR: 1.13) in Model 1, indicating that states are 13 percent more likely to issue a claim in years in which a leadership change occurs. The coefficient  for the lagged leadership change variable in Model 2 is 0.110 (HR: 1.116). Thus, leadership changes in the previous year produce a slightly smaller effect, increasing the hazard rate by 11 percent.
 

% latex table generated in R 3.6.2 by xtable 1.8-4 package
% Wed Oct 14 19:43:31 2020
\begin{table}[htpb]
	\caption{Coalition Change and Territorial Claim Onset \label{tab_s}}
	\centering
	\begin{tabular}{lllll}
		\hline
		& \multicolumn{2}{c}{\underline{Model 1}} & \multicolumn{2}{c}{\underline{Model 2}} \\
		& Structural & Proximate & Structural & Proximate \\ \hline
		Coalition Change$_{t}$      &            & 0.11      &            &          \\
		&            & (0.048)*  &            &          \\
		Coalition Change$_{t-1}$    &            &           &            & 0.162    \\
		&            &           &            & (0.074)* \\
		Independence$_t$            &            & 0.337     &            &          \\
		&            & (0.06)*** &            &          \\
		Independence$_{t-1}$        &            &           &            & 0.167    \\
		&            &           &            & (0.072)* \\
		Regime Transition$_t$       &            & 0.159     &            &          \\
		&            & (0.108)   &            &          \\
		Regime Transition$_{t-1}$   &            &           &            & 0.006    \\
		&            &           &            & (0.105)  \\
		Distance                    & -0.255     &           & -0.242     &          \\
		& (0.034)*** &           & (0.028)*** &          \\
		Previous Territorial Change & 0.091      &           & -0.012     &          \\
		& (0.161)    &           & (0.147)    &          \\
		Former Colony               & 0.74       &           & 0.723      &          \\
		& (0.163)*** &           & (0.215)*** &          \\
		Colonial Contiguity         & 2.263      &           & 2.447      &          \\
		& (0.292)*** &           & (0.266)*** &          \\
		Major-Minor Dyad            & 1.364      &           & 1.443      &          \\
		& (0.134)*** &           & (0.137)*** &          \\
		Major-Major Dyad            & 1.628      &           & 1.848      &          \\
		& (0.475)*** &           & (0.309)*** &          \\
		Defensive Alliance          & -0.215     &           & -0.152     &          \\
		& (0.073)**  &           & (0.089)    &          \\
		Joint Democracy             & -0.51      &           & -0.631     &          \\
		& (0.194)**  &           & (0.188)*** &          \\
		Rivalry                     & 2.076      &           & 2          &          \\
		& (0.132)*** &           & (0.135)*** &          \\
		Intercept                   & -7.299     &           & -7.556     &          \\
		& (0.299)*** &           & (0.254)*** &          \\
		Number of Observations      & 661747     &          & 659438     &         \\
		Number of Failures          & 378        &          & 341        &         \\ \hline
	\end{tabular}
\end{table}

% COALITION CHANGE ---------------------------------------------------------------------------------
Table \ref{tab_s} presents the results of models that include the coalition change variable. Model 1 and Model 2 include variables for coalition change at $t$ and $t-1$ respectively. The estimated coefficient is positive and significant in both models, indicating that coalition changes decrease the time until claim onset. The coefficient in Model 1 and Model 2 are 0.110 (HR: 1.12) and 0.162 (1.180) respectively. Coalition changes are associated with a 12-18 percent increase in the hazard rate.
%  It is worth noting that, contrary to the leadership variables, the substantive effect of the lagged variable is greater than that of the contemporaneous variable. 

% An additional finding emerges when comparing the effects of leadership transitions and coalition changes. By comparing the results in Model 1 and Model 2, we can see that coalition change has a larger substantive effect than that of leadership change. This is consistent with my theory, which predicts that coalition changes generally produce larger changes in the preferences of leaders. By contrast, leadership changes within the context of a coalition change will retain the support of the same backers, which ensures some continuity in policy after the transition.



Although not explicitly theorized, the variables included to account for independence are associated with a significant increase in the hazard rate in three of the four models in Tables \ref{tab_l} and \ref{tab_s}. This indicates that leadership changes are associated with claim onset, which provides support for the argument that changes in the preferences of those who control the levers of power increase the probability of claim onset. This is to expected given the fact that disputes frequently emerge over the jurisdiction of new states. This provides support for the argument that leadership and coalition changes have an effect on claim onset independent of those that accompany the birth of a state. By contrast, the variable for regime transition is not significant in any of the four models. It is possible that this result indicates that institutions are sticky and institutional change takes longer to influence the policies of a state than changes in the preferences of leaders or coalitions.


\subsection{Comparing Cure Models and Case Selection Devices}
In addition to the analyses above, I consider whether the use of case selection methods to estimate standard Cox models produce different inferences from cure models. Table \ref{tab_srd} presents the results of standard Cox proportional hazards in limited samples. The dataset used for Model 1 is limited to politically relevant dyads while Model 2 is limited to a sample of rivals as identified by \citet{colaresi2007}. Contrary to the use of a cure model on the full sample, the estimated coefficients are not statistically significant at the 0.05 level in either model. Thus, using a cure model to control for the probability that claims may potentially emerge produces different results than the use of case selection devices with a standard Cox model.
%The question may emerge of whether this substantive difference emerges because the effects do not hold within the sample or because of bias produced by the cure model. Ultimately the determinant of which model to believe is based on the substantive justification for the model. As noted in Chapter 2 (the methods chapter), the primary determinant of whether to use a cure model or not is whether it makes more sense as a model of the data generating process. 

% Models 1 and 2 are limited to a sample of politically relevant dyads, while Models 3 and 4 are limited to a sample of rivals. 
 
% latex table generated in R 3.6.2 by xtable 1.8-4 package
% Thu Oct 15 01:04:13 2020
\begin{table}[htpb]
	\caption{Cox Models of Coalition Change and Territorial Claim Onset in Politically Relevant Dyads and Rivalries \label{tab_srd}}
	\centering
	\begin{tabular}{lll}
		\hline
									& Model 1    & Model 2	  \\ \hline
		Coalition Change$_{t}$      & 0.078      & 0.524      \\
		                            & (0.155)    & (0.286)    \\
		Independence$_t$            & 2.96       & 2.363      \\
		                            & (0.305)*** & (0.433)*** \\
		Regime Transition$_t$       & 0.569      & 0.016      \\
		                            & (0.238)*   & (0.469)    \\
		Distance                    & -0.2       & -0.208     \\
		                            & (0.038)*** & (0.061)*** \\
		Previous Territorial Change & 0.408      & -0.136     \\
		                            & (0.136)**  & (0.251)    \\
		Former Colony               & 0.576      & 0.823      \\
		                            & (0.201)**  & (0.434)   \\
		Colonial Contiguity         & 0.991      & -0.306     \\
		                            & (0.339)**  & (0.563)    \\
		Major-Minor Dyad            & 0.86       & 1.426      \\
		                            & (0.159)*** & (0.356)*** \\
		Major-Major Dyad            & 1.364      & 1.025      \\
		                            & (0.333)*** & (0.43)*    \\
		Defensive Alliance          & 0.002      & -0.057     \\
		                            & (0.091)    & (0.205)    \\
		Joint Democracy             & -0.36      & -0.362     \\
		                            & (0.191)    & (0.812)    \\
		Rivalry                     & 1.745      &            \\
		                            & (0.136)*** &            \\
		Number of Observations      & 74064      & 2970       \\
		Number of Failures          & 330        & 119        \\ \hline
	\end{tabular}
\end{table}

% latex table generated in R 3.6.2 by xtable 1.8-4 package
% Wed Oct 14 19:43:31 2020
%\begin{table}[ht]
%	\caption{Cox Models of Coalition Change and Territorial Claim Onset in Politically Relevant Dyads and Rivalries \label{tab_srd}}
%	\centering
%	\begin{tabular}{lllll}
%		\hline
%		\hline
%		& Hazard & Hazard & Hazard & Hazard \\ 
%		Coalition Change$_{t}$ & 0.078 &  & 0.524 &  \\ 
%		& (0.155)  &  & (0.286). &  \\ 
%		Coalition Change$_{t-1}$ &  & 0.322 &  & 0.603 \\ 
%		&  & (0.156)* &  & (0.294)* \\ 
%		Independence$_t$ & 2.96 &  & 2.363 &  \\ 
%		& (0.305)*** &  & (0.433)*** &  \\ 
%		Independence$_{t-1}$ &  & 0.797 &  & 0.895 \\ 
%		&  & (0.36)* &  & (0.533). \\ 
%		Regime Transition$_t$ & 0.569 &  & 0.016 &  \\ 
%		& (0.238)* &  & (0.469)  &  \\ 
%		Regime Transition$_{t-1}$ &  & 0.22 &  & -0.147 \\ 
%		&  & (0.265)  &  & (0.51)  \\ 
%		Distance & -0.2 & -0.184 & -0.208 & -0.168 \\ 
%		& (0.038)*** & (0.038)*** & (0.061)*** & (0.06)** \\ 
%		Previous Territorial Change & 0.408 & 0.274 & -0.136 & -0.385 \\ 
%		& (0.136)** & (0.146). & (0.251)  & (0.282)  \\ 
%		Former Colony & 0.576 & 0.502 & 0.823 & 0.908 \\ 
%		& (0.201)** & (0.218)* & (0.434). & (0.482). \\ 
%		Colonial Contiguity & 0.991 & 1.305 & -0.306 & 0.53 \\ 
%		& (0.339)** & (0.344)*** & (0.563)  & (0.66)  \\ 
%		Major-Minor Dyad & 0.86 & 0.936 & 1.426 & 1.448 \\ 
%		& (0.159)*** & (0.165)*** & (0.356)*** & (0.399)*** \\ 
%		Major-Major Dyad & 1.364 & 1.453 & 1.025 & 1.372 \\ 
%		& (0.333)*** & (0.337)*** & (0.43)* & (0.45)** \\ 
%		Defensive Alliance & 0.002 & 0.049 & -0.057 & -0.02 \\ 
%		& (0.091)  & (0.093)  & (0.205)  & (0.204)  \\ 
%		Joint Democracy & -0.36 & -0.475 & -0.362 & -0.315 \\ 
%		& (0.191). & (0.203)* & (0.812)  & (0.827)  \\ 
%		Rivalry & 1.745 & 1.77 &  &  \\ 
%		& (0.136)*** & (0.143)*** &  &  \\ 
%		Number of Observations & 74064 & 73572 & 2970 & 2918 \\ 
%		Number of Failures & 330 & 296 & 119 & 102 \\ 
%		\hline
%	\end{tabular}
%\end{table}

\subsection{Structural Causes of Territorial Claims}
\begin{comment}
distance decreases
terrchange increases
former colony increases
colonial contiguity is posjibve
majorminor dyad increaes
major major increaes
defense null io
jointdem decrease
rivalry increaess

Solschange
terrch insig
defense decreases in m1, not 2

Leadchange
terrch insig
defensive alliance sig in 1, not 2
\end{comment}

Nearly all of the structural variables related to the probability that two states mutually desire the same piece of territory behave as expected across Tables \ref{tab_i} \hyphen \ref{tab_s}. First, the greater the distance between two states, the lower the potential for claims is. States that have a prior colonizer-colony relationship are associated with an increase in the potential for territorial disputes, as are states that have contiguous colonies. Rivals are also more likely to become involved in claims, suggesting that states place a higher value on territory possessed by rival states. Jointly democratic dyads are also less likely to become involved in claims. 

The one issue that receives limited support is previous territorial change. Although it is associated with an increased probability of claims in Table \ref{tab_i}, it is not significant in Tables \ref{tab_l} and \ref{tab_s}. This may be due to the fact that, in some cases, territorial changes occur due to the end of a claim rather than the beginning. This is consistent with the fact that claims that are settled are rarely renewed. The fact that territorial changes are associated with the removal of  from the agenda may thus mask the influence that such changes have on the onset of claims.

The dyadic factors related to opportunity largely behave as expected across all three tables. Dyads that consist of one major power and two major powers are more likely to be involved in claims than dyads consisting of two minor powers. The estimated coefficient for states that are allies is only significant in two models: Model 1 in both Tables \ref{tab_l} and \ref{tab_s}. As such, the results generally do not provide support for the finding that mutual allies are more less likely to become involved in claims.

\section{Discussion and Conclusion}

In this article I sought to answer the question of why leaders begin territorial claims when they do. In many cases states have the potential to justify claims to desirable territory based on history, self-determination, and irredentism, and the location of previous borders, yet choose not to issue claims over them. In theory, leaders have an incentive to issue claims to any potentially valuable territory, as keeping these claims alive may make them easier to justify later. This suggests that they are either incapable of successfully pressuring their opponent into relinquishing their claim to the territory or do not wish to pay the costs of doing so.

Although existing studies demonstrate that a variety of dyadic factors and territorial attributes predict claim onset, they do not consider the question of timing. I proposed that states issue claims when there are changes in either the domestic or international environment that increase the opportunity or willingness of states to press their claims. These findings are robust to the inclusion of a large array of structural variables that predict the potential for claims to emerge between states, thus alleviating the problem associated with the inclusion of many dyads that are not at risk of claim onset.  

The findings of my study are consistent with the idea that changes at the domestic level influence the timing of claim onset. The analysis of domestic variables indicates that changes in a state's leader or winning coalition are associated with an increased probability of claim onset. Previous research indicates that territory that is highly salient to domestic audiences is more likely to become the subject of dispute. My findings complement this research by suggesting that changes in the preferences of the governing elite, and thus the salience of the territory to high-level decisionmakers, affects whether or not states choose to initiate claims over valuable territory. 

% Third, changes in coalition have a larger effect than leadership changes since changes in the winning coalition generally produce larger changes in preferences than leadership changes.

I also find support for the argument that changes at the international level influence claim onset. Although short-term changes in capabilities between claimants are not associated with claim onset, the occurrence of militarized disputes is. I also find that states were more likely to initiate claims during periods in which the international system underwent rapid changes, including the era surrounding German and Italian unification, World War I, and World War II.

% Structural Model	

Generally speaking, my results indicate that the structural factors that influence the potential for claims to arise have effects consistent with previous research. I find that factors that increase the probability that two states desire the same territory and factors that provide them with the opportunity to issue claims influence the onset of claims. One finding arises from the structural model that has generally not been explored by the literature. Prior research has not examined whether rivalry or a history of militarized competition influences the probability of claim onset (\citet{rasler2006} is an exception, although they do not test this argument explicitly. I find that rivals are much more likely to become involved in territorial claims. This finding has potential implications for theories connecting territorial claims, power politics, and rivalry \citep{senese2008, vasquez2009}. Generally, previous work assumes that territorial claims increase the probability that states engage in power politics tactics which leads to the onset of rivalry. However, the results with respect to rivalry suggest that territorial claims are more likely to emerge between hostile states to begin with. Although territorial claims undoubtedly increase the probability of rivalry, it is worth reexamining the sequence of events that lead to claim onset and subsequent militarized disputes, as it is possible that these states are more prone to the use of power politics to begin with.

% Diverse Pathways

Future research may also consider whether the structural and proximate causes of claims interact with each other. In this paper, I assume that the proximate causes of claim onset have an equal effect once all structural factors have been controlled for. However, it is possible that some proximate causes may only have an effect in the presence of specific structural conditions. For example, as noted above, states may be more likely to issue claims over economically or strategically valuable territory if they have a history of conflictual interactions with each other. Similarly, claims that are closely tied to identity politics may be more likely among states that have certain types of domestic regimes. Future research should work to uncover whether there are distinct pathways to different types of territorial claims and work to specify these more fully.

% However, it is possible that both the structural and proximate causes of claim onset differ depending on the type of territory contested.

% Consider when power dynamics matter: Wars of rivalry vs wars of inequality – Vasquez, dougs paper, david and goliath

% Implication: renewed claims – unpack finding on lagterrch

%\pagebreak
%\section{References}
%\printbibliography[heading=none]

% \end{document}
% PRD Cites
% Lemke and Reed 2001, Russett and Oneal 2001
